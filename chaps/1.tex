\chapter{Introduction}
Detecting objects of interest from a complex scene is a basic perceptual skill in human beings and other animals.
In computer science, object detection~\citep{dod} is a computer technology that deals with detecting instances of semantic objects of a certain class (such as humans, buildings, or cars) in digital images and videos. And successful object detection
methods play fundamental roles in many application areas, which include video surveillance, driving assistance, image retrieval, etc.

Most modern detection methods fall into two categories. Some~\citep{ij4,ac31,ac30,ac4,ac32,ac29,ac28,ac1} follow the sliding-window schema, and they detect objects by consider whether each of the sub-images contains an instance of the target object. The other methods~\citep{ac9,ac2,ac3,ac22,lb1,ac5,ac10,ac21,ac18} infer object centers based on local image features in a bottom-up manner. These methods start with detection of object parts, in the form of image patches, edgelets, or keypoints, and then make inferences about the target objects' states, like position, or label.


 Humans still far outperform computers in the tasks of image-based recognition and detection. Only a few techniques are mature enough for daily applications, i.e., face detection~\citep{face} used in cameras. For years, many researchers in the area of computer vision focus on object detection from images. Their efforts include proposing better image features~\citep{o17} or better models for object representation~\citep{bgf}, proposing better discriminative classifiers~\citep{dlearn} or better inference model~\citep{hdp}, or proposing better searching techniques for exploring solution space~\citep{bab}.

 In this thesis, efforts are also made to improve performance of detection methods. These efforts try to explore how to use the information which previous methods do not make full use of. Roughly, the efforts belong to two categories, the first category is exploring approaches of efficiently and effectively combing of motion information with appearance information, and the second category is exploring  how to combine visual and spatial information encoded in local image features of the same object.

For fusion of information from different channels, voting systems are employed. Voting is preferred  for its robustness in using local information, and its inference procedure's capability to use global information.

  Many methods for detection mainly use either appearance information or motion information. Following some previous work~\citep{pvm},  the first two methods will address that when well combining the information from these two channels, detection performance will be better.


 The first method is developed mainly for real-time applications under limited computational power.  This method can be considered as a three-step method. The first step deals with keypoints. It
takes original data as input, and outputs keypoint clusters as detection hypotheses. This step detects, verifies, and clusters keypoints. The second
step takes these keypoint clusters as input, verifies them by their appearance and motion
information, and outputs the ones which pass verifications as detection results. The last step feeds the detection results from step two into a voting system. Since detection results are connected by their belonging trajectories, voting along the temporal dimension is responsible for giving the final decision of  each object,  when it disappears from the scene.   Motion information plays a very important role in the method. The target objects are considered as possessing both particular appearance patterns and motion patterns. When the second step verifies the detection hypotheses using appearance information, a biased classifier is used. This classifier produces more false alarms to pursue higher detection rate. Then motion information is used to filter out the false alarms. Motion information in the form of trajectories also connects weak inferences and feeds the weak inferences into a voting system for the final results. In addition, the pipeline of this method is optimized in a hierarchical way. In the pipeline, the later one step is, the more time-consuming it is, and the fewer instances it will deal with. The method performs well under simple scene, i.e., data collected by infrared cameras in a tunnel environment, and gives promising detection results in the experiments. However, the performance of this method under complicated scene is not promising. And then we propose the second method.

The second method belongs to methods based on Hough transform. It extend the Implicit Shape Model~\citep{lb1} to combine motion information. For training, image features together with labels and offsets to object centers of sample images are considered as codes, and inserted into a codebook. For detection, image features are detected on the target image, and then matched against the codebook using image feature as key. The matched codes will indicate the labels and object centers. During the detection step, this method firstly do motion analysis, which results in grouping results of the image features on the target image. The grouping results are used during the inference for labels and centers of the target objects. It is assumed that image features with the same motion pattern, here in the same motion group, should belong to the same object. The inference procedure then prefers the label and position inferences with more consistence in the same motion group. On two datasets, the proposed method outperforms the state-of-the-art method.

While the second method performs well under complicated scene, it is relatively slow. This is due to the time-consuming property of methods based on Hough transform.   The third method aims at improving the efficiency of the second method. Also it tries to flatten the gap of appearance and positional information. This method does not use motion information. In methods based on Hough transform, image features are used as key to query similar codes from the codebook, and in the third method, both appearance and position are used as key. The bottom-up property of Hough transform also ignore the relationship between different image features. Actually, the mutual information encoded in the image features of the same object is very informational. The third method considers objects as point sets of, i.e., of 12-dimensional, while the first 10 dimensions are appearance information, and the last 2 dimensions are positional information. The training step is almost the same with Hough-transform methods, except for how a few parameters are trained. At the detection step, instead of using the appearance information of one single feature for querying, the point set of a sub-image is used for querying. Pyramid Matching is used for accelerating the querying. The procedure ensures the full use of the visual and spatial  information encoded in the image features of the same object. While giving promising detection results on two datasets, this method is confirmed to be much more efficient than the second method.

The thesis is organised as follows. Chapter \ref{chp2} gives background and reviews some related work. Chapter \ref{chp3} introduces the method aimed at efficient detection by combining motion and appearance information. Chapter \ref{chp4} proposes the method that extends the Implicit Shape Model to incorporate motion information, and the method groups object parts for detection. Chapter \ref{chp5} presents the method which detects by Pyramid Match Score. Chapter \ref{chp6} concludes, and discusses about possible improvements of the proposed methods for future work.



